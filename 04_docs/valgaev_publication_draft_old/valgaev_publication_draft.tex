% !TeX program = pdfLaTeX
\documentclass[smallextended]{svjour3}       % onecolumn (second format)
%\documentclass[twocolumn]{svjour3}          % twocolumn
%
\smartqed  % flush right qed marks, e.g. at end of proof
%
\usepackage{amsmath}
\usepackage{graphicx}
\usepackage[utf8]{inputenc}

\usepackage[hyphens]{url} % not crucial - just used below for the URL
\usepackage{hyperref}

%
% \usepackage{mathptmx}      % use Times fonts if available on your TeX system
%
% insert here the call for the packages your document requires
%\usepackage{latexsym}
% etc.
%
% please place your own definitions here and don't use \def but
% \newcommand{}{}
%
% Insert the name of "your journal" with
% \journalname{myjournal}
%

%% load any required packages here



% tightlist command for lists without linebreak
\providecommand{\tightlist}{%
  \setlength{\itemsep}{0pt}\setlength{\parskip}{0pt}}



\usepackage{booktabs}
\usepackage{longtable}
\usepackage{array}
\usepackage{multirow}
\usepackage{wrapfig}
\usepackage{float}
\usepackage{colortbl}
\usepackage{pdflscape}
\usepackage{tabu}
\usepackage{threeparttable}
\usepackage{threeparttablex}
\usepackage[normalem]{ulem}
\usepackage{makecell}
\usepackage{xcolor}
\begin{document}


\title{Title here \thanks{Grants or other notes about the article that
should go on the front page should be placed here. General
acknowledgments should be placed at the end of the article.} }
 \subtitle{Do you have a subtitle? If so, write it here} 

    \titlerunning{Short form of title (if too long for head)}

\author{  Äüthör 1 \and  Âuthóř 2 \and  }

    \authorrunning{ Short form of author list if too long for running
head }

\institute{
        Äüthör 1 \at
     Department of YYY, University of XXX \\
     \email{\href{mailto:abc@def}{\nolinkurl{abc@def}}}  %  \\
%             \emph{Present address:} of F. Author  %  if needed
    \and
        Âuthóř 2 \at
     Department of ZZZ, University of WWW \\
     \email{\href{mailto:djf@wef}{\nolinkurl{djf@wef}}}  %  \\
%             \emph{Present address:} of F. Author  %  if needed
    \and
    }

\date{Received: date / Accepted: date}
% The correct dates will be entered by the editor


\maketitle

\begin{abstract}
The text of your abstract. 150 -- 250 words.
\\
\keywords{
        key \and
        dictionary \and
        word \and
    }

    \subclass{
                    MSC code 1 \and
                    MSC code 2 \and
            }

\end{abstract}


\def\spacingset#1{\renewcommand{\baselinestretch}%
{#1}\small\normalsize} \spacingset{1}


\hypertarget{intro}{%
\section{Introduction}\label{intro}}

Your text comes here. Separate text sections with \cite{Mislevy06Cog}.

\hypertarget{sec:1}{%
\section{Section title}\label{sec:1}}

Text with citations by \cite{Galyardt14mmm}.

\hypertarget{sec:2}{%
\subsection{Subsection title}\label{sec:2}}

as required. Don't forget to give each section and subsection a unique
label (see Sect. \ref{sec:1}).

\hypertarget{paragraph-headings}{%
\paragraph{Paragraph headings}\label{paragraph-headings}}

Use paragraph headings as needed.

\begin{align}
a^2+b^2=c^2
\end{align}

\hypertarget{instructions}{%
\section{---Instructions---}\label{instructions}}

\begin{itemize}
\item
  Headers are specified with \#, \#\#, \#\#\#, etc. for H1, H2, H3, etc.
\item
  When you would like to cite a paper, your citation looks like this:
  ``(\textbf{Key?})'' - e.g. (\textbf{author2017?})
\item
  If you would like to add a figure, this can be done with an R snippet:
\item
  To cross-reference figures in your text, simply write something like:
  As shown in Figure @ref(fig:figRef) \ldots{}
\item
  Tables can be added using the kable function from knitr, for example:
\item
  Cross-referencing a Table is similar to that of a Figure. For example:
  Table @ref(tab:tabRef) shows the data\ldots{}
\item
  Put \#References at the end of your document to generate the
  bibliography.
\item
  And don't forget to have a look here for more resources:
  \url{http://rmarkdown.rstudio.com/}
\end{itemize}

More info:
\url{http://landscapeportal.org/blog/2017/09/06/r-markdown-template-for-a-scientific-manuscript/}


\bibliographystyle{spphys}
\bibliography{bibliography.bib}


\end{document}
